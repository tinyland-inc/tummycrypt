\documentclass[11pt,a4paper]{article}
% preamble.tex — Shared LaTeX preamble for all tcfs design documents
% Usage: % preamble.tex — Shared LaTeX preamble for all tcfs design documents
% Usage: % preamble.tex — Shared LaTeX preamble for all tcfs design documents
% Usage: \input{preamble} at top of each .tex file

% Encoding and fonts
\usepackage[T1]{fontenc}
\usepackage[utf8]{inputenc}
\usepackage{lmodern}
\usepackage{textcomp}

% Page geometry
\usepackage[margin=2.5cm]{geometry}

% Colors and hyperlinks
\usepackage[dvipsnames]{xcolor}
\usepackage[colorlinks=true,linkcolor=NavyBlue,urlcolor=NavyBlue,citecolor=NavyBlue]{hyperref}

% Code listings
\usepackage{listings}
\lstset{
  basicstyle=\ttfamily\small,
  breaklines=true,
  frame=single,
  backgroundcolor=\color{gray!10},
  xleftmargin=2em,
  framexleftmargin=1.5em,
  numbers=none,
  showstringspaces=false,
  columns=flexible,
  literate={←}{$\leftarrow$}1 {→}{$\rightarrow$}1 {⟶}{$\longrightarrow$}1,
}
\lstdefinelanguage{json}{
  string=[s]{"}{"},
  comment=[l]{//},
  morecomment=[s]{/*}{*/},
  literate=
    *{:}{{{\color{NavyBlue}:}}}{1}
    {,}{{{\color{NavyBlue},}}}{1}
    {\{}{{{\color{NavyBlue}\{}}}{1}
    {\}}{{{\color{NavyBlue}\}}}}{1}
    {[}{{{\color{NavyBlue}[}}}{1}
    {]}{{{\color{NavyBlue}]}}}{1},
}
\lstdefinelanguage{toml}{
  comment=[l]{\#},
  string=[b]",
  morestring=[b]',
}
\lstdefinelanguage{protobuf}{
  morekeywords={service,rpc,returns,message,string,uint32,uint64,int64,bool,bytes,repeated,optional,oneof,enum,map},
  morecomment=[l]{//},
  morestring=[b]",
}

% Tables
\usepackage{longtable}
\usepackage{booktabs}
\usepackage{array}
\usepackage{tabularx}

% Graphics
\usepackage{graphicx}
\usepackage{float}

% Better lists
\usepackage{enumitem}

% Header/footer
\usepackage{fancyhdr}
\pagestyle{fancy}
\fancyhf{}
\fancyfoot[C]{\thepage}
\renewcommand{\headrulewidth}{0.4pt}

% Tighter spacing
\setlength{\parindent}{0pt}
\setlength{\parskip}{6pt}

% Convenience macros
\newcommand{\tcfs}{\texttt{tcfs}}
\newcommand{\tcfsd}{\texttt{tcfsd}}
\newcommand{\code}[1]{\texttt{#1}}
\newcommand{\filepath}[1]{\texttt{#1}}
 at top of each .tex file

% Encoding and fonts
\usepackage[T1]{fontenc}
\usepackage[utf8]{inputenc}
\usepackage{lmodern}
\usepackage{textcomp}

% Page geometry
\usepackage[margin=2.5cm]{geometry}

% Colors and hyperlinks
\usepackage[dvipsnames]{xcolor}
\usepackage[colorlinks=true,linkcolor=NavyBlue,urlcolor=NavyBlue,citecolor=NavyBlue]{hyperref}

% Code listings
\usepackage{listings}
\lstset{
  basicstyle=\ttfamily\small,
  breaklines=true,
  frame=single,
  backgroundcolor=\color{gray!10},
  xleftmargin=2em,
  framexleftmargin=1.5em,
  numbers=none,
  showstringspaces=false,
  columns=flexible,
  literate={←}{$\leftarrow$}1 {→}{$\rightarrow$}1 {⟶}{$\longrightarrow$}1,
}
\lstdefinelanguage{json}{
  string=[s]{"}{"},
  comment=[l]{//},
  morecomment=[s]{/*}{*/},
  literate=
    *{:}{{{\color{NavyBlue}:}}}{1}
    {,}{{{\color{NavyBlue},}}}{1}
    {\{}{{{\color{NavyBlue}\{}}}{1}
    {\}}{{{\color{NavyBlue}\}}}}{1}
    {[}{{{\color{NavyBlue}[}}}{1}
    {]}{{{\color{NavyBlue}]}}}{1},
}
\lstdefinelanguage{toml}{
  comment=[l]{\#},
  string=[b]",
  morestring=[b]',
}
\lstdefinelanguage{protobuf}{
  morekeywords={service,rpc,returns,message,string,uint32,uint64,int64,bool,bytes,repeated,optional,oneof,enum,map},
  morecomment=[l]{//},
  morestring=[b]",
}

% Tables
\usepackage{longtable}
\usepackage{booktabs}
\usepackage{array}
\usepackage{tabularx}

% Graphics
\usepackage{graphicx}
\usepackage{float}

% Better lists
\usepackage{enumitem}

% Header/footer
\usepackage{fancyhdr}
\pagestyle{fancy}
\fancyhf{}
\fancyfoot[C]{\thepage}
\renewcommand{\headrulewidth}{0.4pt}

% Tighter spacing
\setlength{\parindent}{0pt}
\setlength{\parskip}{6pt}

% Convenience macros
\newcommand{\tcfs}{\texttt{tcfs}}
\newcommand{\tcfsd}{\texttt{tcfsd}}
\newcommand{\code}[1]{\texttt{#1}}
\newcommand{\filepath}[1]{\texttt{#1}}
 at top of each .tex file

% Encoding and fonts
\usepackage[T1]{fontenc}
\usepackage[utf8]{inputenc}
\usepackage{lmodern}
\usepackage{textcomp}

% Page geometry
\usepackage[margin=2.5cm]{geometry}

% Colors and hyperlinks
\usepackage[dvipsnames]{xcolor}
\usepackage[colorlinks=true,linkcolor=NavyBlue,urlcolor=NavyBlue,citecolor=NavyBlue]{hyperref}

% Code listings
\usepackage{listings}
\lstset{
  basicstyle=\ttfamily\small,
  breaklines=true,
  frame=single,
  backgroundcolor=\color{gray!10},
  xleftmargin=2em,
  framexleftmargin=1.5em,
  numbers=none,
  showstringspaces=false,
  columns=flexible,
  literate={←}{$\leftarrow$}1 {→}{$\rightarrow$}1 {⟶}{$\longrightarrow$}1,
}
\lstdefinelanguage{json}{
  string=[s]{"}{"},
  comment=[l]{//},
  morecomment=[s]{/*}{*/},
  literate=
    *{:}{{{\color{NavyBlue}:}}}{1}
    {,}{{{\color{NavyBlue},}}}{1}
    {\{}{{{\color{NavyBlue}\{}}}{1}
    {\}}{{{\color{NavyBlue}\}}}}{1}
    {[}{{{\color{NavyBlue}[}}}{1}
    {]}{{{\color{NavyBlue}]}}}{1},
}
\lstdefinelanguage{toml}{
  comment=[l]{\#},
  string=[b]",
  morestring=[b]',
}
\lstdefinelanguage{protobuf}{
  morekeywords={service,rpc,returns,message,string,uint32,uint64,int64,bool,bytes,repeated,optional,oneof,enum,map},
  morecomment=[l]{//},
  morestring=[b]",
}

% Tables
\usepackage{longtable}
\usepackage{booktabs}
\usepackage{array}
\usepackage{tabularx}

% Graphics
\usepackage{graphicx}
\usepackage{float}

% Better lists
\usepackage{enumitem}

% Header/footer
\usepackage{fancyhdr}
\pagestyle{fancy}
\fancyhf{}
\fancyfoot[C]{\thepage}
\renewcommand{\headrulewidth}{0.4pt}

% Tighter spacing
\setlength{\parindent}{0pt}
\setlength{\parskip}{6pt}

% Convenience macros
\newcommand{\tcfs}{\texttt{tcfs}}
\newcommand{\tcfsd}{\texttt{tcfsd}}
\newcommand{\code}[1]{\texttt{#1}}
\newcommand{\filepath}[1]{\texttt{#1}}


\fancyhead[L]{\small\textsc{tcfs}}
\fancyhead[R]{\small FileProvider Design}

\title{tcfs FileProvider Integration\\
\large macOS \& iOS On-Demand Hydration via NSFileProviderReplicatedExtension}
\author{tcfs / TummyCrypt}
\date{February 2026}

\begin{document}
\maketitle
\tableofcontents
\newpage

% ─────────────────────────────────────────────────────────────────────────────
\section{Overview}

This document describes the design for integrating \tcfs{} with Apple's
FileProvider framework on macOS and iOS. FileProvider is the official mechanism
for cloud storage providers to present files in Finder (macOS) and the Files app
(iOS) with on-demand hydration, cloud status icons, and native Spotlight
integration.

FileProvider replaces FUSE as the primary file integration point on Apple
platforms. FUSE remains the correct approach for Linux via \code{tcfs-fuse}.

\subsection{Design Goals}

\begin{enumerate}
  \item \textbf{Native Finder integration} --- files appear under Locations in
        the Finder sidebar at \filepath{\textasciitilde/Library/CloudStorage/}.
  \item \textbf{On-demand hydration} --- files exist as APFS dataless stubs
        until opened; kernel calls FileProvider extension to fetch content.
  \item \textbf{Reuse existing Rust crates} --- \code{tcfs-storage},
        \code{tcfs-chunks}, \code{tcfs-crypto}, and \code{tcfs-sync} are
        linked as a static library with a thin Swift shim.
  \item \textbf{No kernel extensions} --- FileProvider runs entirely in
        userspace as an application extension (.appex).
  \item \textbf{No root privileges} --- no \code{sudo}, no system extension
        approval, no Reduced Security mode on Apple Silicon.
  \item \textbf{Cross-Apple platform} --- shared Rust crate serves both
        macOS and iOS with platform-conditional compilation.
\end{enumerate}

% ─────────────────────────────────────────────────────────────────────────────
\section{Platform Integration Matrix}

\begin{table}[H]
\centering
\begin{tabularx}{\textwidth}{lllXl}
\toprule
\textbf{Platform} & \textbf{Framework} & \textbf{Crate} & \textbf{Mechanism} & \textbf{Status} \\
\midrule
Linux   & FUSE3                & tcfs-fuse         & Kernel FUSE module           & Working (v0.3.0) \\
Windows & Cloud Files API      & tcfs-cloudfilter  & CFAPI minifilter             & Skeleton \\
macOS   & FileProvider         & tcfs-fileprovider & APFS dataless files          & This design \\
iOS     & FileProvider         & tcfs-fileprovider & APFS dataless files          & RFC 0003 \\
\bottomrule
\end{tabularx}
\caption{Per-platform file integration strategy}
\end{table}

All four platforms share the same hydration concept: a lightweight stub
represents each remote file; content is fetched on demand from SeaweedFS via
the CAS chunk store.

% ─────────────────────────────────────────────────────────────────────────────
\section{Architecture}

\subsection{Component Stack}

\begin{lstlisting}[language={},title={FileProvider component stack}]
[Finder / CLI / Any App]
       |
       v
[macOS Kernel -- APFS dataless files]
       |  (on-demand hydration callback)
       v
[FileProvider Extension (.appex)]     ~200 LOC Swift
       |  extern "C" function calls
       v
[tcfs-fileprovider (Rust static lib)]
       |  uses existing tcfs crates
       +---> tcfs-storage  (OpenDAL S3)
       +---> tcfs-chunks   (FastCDC + BLAKE3 + zstd)
       +---> tcfs-crypto   (XChaCha20-Poly1305)
       +---> tcfs-sync     (vector clocks, state cache)
       |
       v
[SeaweedFS S3 / NATS JetStream]
\end{lstlisting}

\subsection{Extension Bundle Structure}

FileProvider extensions are delivered inside a host application:

\begin{lstlisting}[language={}]
TummyCrypt.app/
  Contents/
    MacOS/
      TummyCrypt            # Host app (manages extension lifecycle)
    PlugIns/
      TcfsFileProvider.appex/
        Contents/
          MacOS/
            TcfsFileProvider   # Extension binary (Swift + Rust staticlib)
          Info.plist           # NSExtension metadata
          Resources/
            tcfs_fileprovider.h  # C header (cbindgen output)
    Frameworks/
      libtcfs_fileprovider.a   # Rust static library
\end{lstlisting}

\subsection{Data Flow}

\begin{enumerate}
  \item \textbf{Enumeration}: Finder opens the tcfs location. The extension's
        \code{enumerator(for:)} creates an enumerator that calls the Rust
        backend's \code{tcfs\_fp\_enumerate\_dir()}, which lists manifests
        from S3. Each manifest becomes an \code{NSFileProviderItem} with
        metadata (name, size, hash, modified time).

  \item \textbf{Hydration}: User opens a cloud-only file. The kernel detects
        the APFS dataless flag and calls \code{fetchContents(for:version:)}.
        The Swift shim calls \code{tcfs\_fp\_fetch\_contents()}, which:
        \begin{enumerate}[label=\alph*)]
          \item Fetches the manifest from \code{manifests/\{file\_hash\}}
          \item Fetches chunks in parallel from \code{chunks/\{chunk\_hash\}}
          \item Decrypts each chunk (XChaCha20-Poly1305 per-chunk encryption)
          \item Decompresses each chunk (zstd)
          \item Reassembles and writes to the provided temporary URL
        \end{enumerate}
        The kernel then delivers the content to the requesting application.

  \item \textbf{Upload}: User saves a file. The extension's \code{createItem}
        or \code{modifyItem} callback receives the new content. The Rust
        backend chunks, compresses, encrypts, and pushes to S3.

  \item \textbf{Eviction}: When disk space is low, macOS automatically evicts
        hydrated files back to dataless stubs. The extension's
        \code{materializedSetDidChange()} callback is notified.

  \item \textbf{Sync}: Background sync is triggered by
        \code{NSFileProviderManager.signalEnumerator(for:)}, which causes
        the system to re-enumerate and detect remote changes.
\end{enumerate}

% ─────────────────────────────────────────────────────────────────────────────
\section{Swift Shim Design}

The FileProvider extension entry point must be Swift or Objective-C.
The shim implements three protocols with all logic delegated to Rust:

\subsection{Extension Entry Point}

\begin{lstlisting}[language={},title={TcfsFileProvider.swift (simplified)}]
import FileProvider

class TcfsFileProvider: NSObject,
    NSFileProviderReplicatedExtension {

    let domain: NSFileProviderDomain

    required init(domain: NSFileProviderDomain) {
        self.domain = domain
        tcfs_fp_initialize()  // Rust: load config, connect S3
    }

    func invalidate() {
        tcfs_fp_shutdown()
    }

    func item(for identifier: NSFileProviderItemIdentifier,
              request: NSFileProviderRequest,
              completionHandler: @escaping (NSFileProviderItem?,
                  Error?) -> Void) -> Progress {
        // Call Rust backend
        var item = FPItem()
        let rc = tcfs_fp_get_item_metadata(
            identifier.rawValue, &item)
        if rc == 0 {
            completionHandler(TcfsItem(item), nil)
        } else {
            completionHandler(nil, mapError(rc))
        }
        return Progress()
    }

    func fetchContents(
        for itemIdentifier: NSFileProviderItemIdentifier,
        version: NSFileProviderItemVersion?,
        request: NSFileProviderRequest,
        completionHandler: @escaping (URL?,
            NSFileProviderItem?, Error?) -> Void
    ) -> Progress {
        let progress = Progress(totalUnitCount: 100)
        DispatchQueue.global().async {
            let tmpURL = /* temporary file URL */
            let rc = tcfs_fp_fetch_contents(
                itemIdentifier.rawValue,
                tmpURL.path,
                { pct in progress.completedUnitCount =
                    Int64(pct * 100) }
            )
            if rc == 0 {
                completionHandler(tmpURL, /*item*/, nil)
            } else {
                completionHandler(nil, nil, mapError(rc))
            }
        }
        return progress
    }

    func enumerator(
        for containerItemIdentifier:
            NSFileProviderItemIdentifier,
        request: NSFileProviderRequest
    ) throws -> NSFileProviderEnumerator {
        return TcfsEnumerator(containerItemIdentifier)
    }
}
\end{lstlisting}

\subsection{Item and Enumerator}

The \code{TcfsItem} class wraps the C struct returned by Rust:

\begin{lstlisting}[language={},title={C bridge types (cbindgen output)}]
typedef struct {
    const char *item_id;
    const char *filename;
    const char *parent_id;
    uint64_t    file_size;
    int64_t     modified_ts;
    bool        is_directory;
    const char *content_hash;
} FPItem;
\end{lstlisting}

The enumerator calls \code{tcfs\_fp\_enumerate\_dir()} and wraps each
\code{FPItem} in an \code{NSFileProviderItem}-conforming Swift object.

% ─────────────────────────────────────────────────────────────────────────────
\section{Rust Static Library}

\subsection{Crate Configuration}

\begin{lstlisting}[language=toml,title={crates/tcfs-fileprovider/Cargo.toml (future)}]
[package]
name = "tcfs-fileprovider"
version.workspace = true
edition.workspace = true

[lib]
crate-type = ["staticlib"]

[dependencies]
tcfs-core    = { path = "../tcfs-core" }
tcfs-storage = { path = "../tcfs-storage" }
tcfs-chunks  = { path = "../tcfs-chunks" }
tcfs-crypto  = { path = "../tcfs-crypto" }
tcfs-sync    = { path = "../tcfs-sync" }
tokio = { workspace = true }
anyhow = { workspace = true }
tracing = { workspace = true }

[build-dependencies]
cbindgen = "0.27"
\end{lstlisting}

\subsection{C API Surface}

The Rust library exposes a minimal C API. All functions return \code{i32}
error codes (0 = success, negative = error) and write output through
pointer parameters.

\begin{lstlisting}[language={},title={Exported C functions}]
// Lifecycle
int32_t tcfs_fp_initialize(void);
int32_t tcfs_fp_shutdown(void);

// Enumeration
int32_t tcfs_fp_enumerate_dir(
    const char *path,
    FPItem    **out_items,
    size_t     *out_count
);
void tcfs_fp_free_items(FPItem *items, size_t count);

// Item metadata
int32_t tcfs_fp_get_item_metadata(
    const char *item_id,
    FPItem     *out_item
);

// Hydration (download + decrypt + decompress)
int32_t tcfs_fp_fetch_contents(
    const char *item_id,
    const char *dest_path,
    void      (*progress_cb)(double)
);

// Upload (chunk + compress + encrypt + push)
int32_t tcfs_fp_upload_file(
    const char *local_path,
    const char *remote_path,
    void      (*progress_cb)(double)
);

// Sync status
int32_t tcfs_fp_sync_status(FPSyncStatus *out);
\end{lstlisting}

\subsection{Async Bridge}

The Rust library owns a \code{tokio::Runtime} for async operations. All
C-exported functions block on the runtime:

\begin{lstlisting}[language={},title={Async bridge pattern}]
use std::sync::LazyLock;
use tokio::runtime::Runtime;

static RUNTIME: LazyLock<Runtime> = LazyLock::new(|| {
    Runtime::new().expect("Failed to create tokio runtime")
});

#[no_mangle]
pub extern "C" fn tcfs_fp_fetch_contents(
    item_id: *const c_char,
    dest: *const c_char,
    progress: extern "C" fn(f64),
) -> i32 {
    RUNTIME.block_on(async {
        // ... hydration using tcfs-storage, tcfs-chunks,
        //     tcfs-crypto (all async)
    })
}
\end{lstlisting}

% ─────────────────────────────────────────────────────────────────────────────
\section{Credential Storage}

FileProvider extensions run in a sandboxed process and cannot read environment
variables or arbitrary config files. Credentials are stored via:

\begin{description}
  \item[Keychain] S3 access and secret credentials stored in a shared Keychain
    access group (\code{group.com.tummycrypt.tcfs}). The host app writes
    credentials during setup; the extension reads them at runtime.
  \item[App Group UserDefaults] Configuration values (S3 endpoint, bucket,
    prefix, NATS URL) stored in shared UserDefaults.
  \item[App Group Container] Large config files or state caches stored in
    the shared App Group container directory.
\end{description}

The credential flow:

\begin{enumerate}
  \item User launches TummyCrypt.app host application
  \item Host app prompts for S3 endpoint, credentials, encryption passphrase
  \item Credentials written to shared Keychain; config to shared UserDefaults
  \item FileProvider extension reads credentials on first enumeration
  \item Encryption key derived from passphrase via Argon2id (same as desktop)
\end{enumerate}

% ─────────────────────────────────────────────────────────────────────────────
\section{Build Pipeline}

\subsection{Cross-Compilation}

\begin{lstlisting}[language={},title={Build steps}]
# 1. Build Rust static library for Apple targets
cargo build --lib -p tcfs-fileprovider \
    --target aarch64-apple-darwin    # macOS Apple Silicon
cargo build --lib -p tcfs-fileprovider \
    --target aarch64-apple-ios       # iOS

# 2. Generate C header
cbindgen --crate tcfs-fileprovider --lang c \
    -o TcfsFileProvider.h

# 3. Build Xcode project (extension + host app)
xcodebuild -project TummyCrypt.xcodeproj \
    -scheme TummyCrypt \
    -configuration Release \
    -archivePath build/TummyCrypt.xcarchive \
    archive

# 4. Sign and notarize
codesign --sign "Developer ID Application: Tinyland Inc" \
    --options runtime \
    --entitlements entitlements.plist \
    build/TummyCrypt.app

xcrun notarytool submit build/TummyCrypt.zip \
    --apple-id "$APPLE_ID" \
    --team-id "$TEAM_ID" \
    --password "$APP_SPECIFIC_PASSWORD" \
    --wait
\end{lstlisting}

\subsection{CI Integration}

The build runs on macOS GitHub Actions runners:

\begin{enumerate}
  \item Rust cross-compilation: \code{cargo build} with Apple targets
  \item \code{cbindgen} header generation
  \item \code{xcodebuild} for the extension bundle
  \item Code signing with stored certificates (GitHub Secrets)
  \item Notarization via \code{notarytool}
  \item Archive as \code{.dmg} release asset
\end{enumerate}

% ─────────────────────────────────────────────────────────────────────────────
\section{Comparison with Other Platforms}

\subsection{Hydration Trigger Comparison}

\begin{table}[H]
\centering
\begin{tabularx}{\textwidth}{lXXX}
\toprule
\textbf{Concept} & \textbf{Linux (tcfs-fuse)} & \textbf{Windows (tcfs-cloudfilter)} & \textbf{macOS/iOS (tcfs-fileprovider)} \\
\midrule
Stub file        & \code{.tc} JSON file         & CFAPI placeholder         & APFS dataless file \\
Hydration trigger & \code{read()} syscall        & \code{CF\_CALLBACK\_FETCH\_DATA} & \code{fetchContents(for:)} \\
Dehydration      & \code{unsync} command         & \code{CfDehydratePlaceholder}    & System automatic eviction \\
Directory listing & \code{readdir()}             & \code{CfGetPlaceholders}          & \code{enumerator(for:)} \\
Status icons     & None (FUSE limitation)        & Explorer cloud icons              & Finder cloud icons \\
Spotlight        & N/A                           & Windows Search integration        & Native APFS indexing \\
\bottomrule
\end{tabularx}
\caption{Platform comparison for file integration}
\end{table}

\subsection{Reuse Matrix}

\begin{table}[H]
\centering
\begin{tabular}{lcccc}
\toprule
\textbf{Crate} & \textbf{Linux} & \textbf{Windows} & \textbf{macOS} & \textbf{iOS} \\
\midrule
tcfs-core       & 100\% & 100\% & 100\% & 100\% \\
tcfs-storage    & 100\% & 100\% & 100\% & 70\%  \\
tcfs-chunks     & 100\% & 100\% & 100\% & 100\% \\
tcfs-crypto     & 100\% & 100\% & 100\% & 100\% \\
tcfs-sync       & 100\% & 100\% & 100\% & 80\%  \\
\bottomrule
\end{tabular}
\caption{Crate reuse across platforms (iOS storage/sync need minor adaptation)}
\end{table}

% ─────────────────────────────────────────────────────────────────────────────
\section{Implementation Phases}

\begin{description}
  \item[Phase 1 (v0.4.x)] Transitional: macOS daemon with \code{--no-mount}
    default. Push/pull/sync work without FUSE. macFUSE opt-in for existing users.
  \item[Phase 2 (v0.5.x)] FileProvider MVP: read-only enumeration + on-demand
    hydration. Rust staticlib + Swift shim. Keychain credentials. Apple
    Developer ID enrollment.
  \item[Phase 3 (v0.6.x)] Full FileProvider: write support, conflict resolution,
    background sync, progress reporting, file pinning, E2E encryption through
    the hydration path.
  \item[Phase 4 (v1.0)] Polish: automatic eviction integration, thumbnails,
    QuickLook, share extension. Deprecate macFUSE transitional path.
\end{description}

% ─────────────────────────────────────────────────────────────────────────────
\section{Open Questions}

\begin{enumerate}
  \item \textbf{Apple Developer enrollment}: Is Tinyland Inc enrolled in the
    Apple Developer Program (\$99/year)? If not, who initiates enrollment?
  \item \textbf{App Group identifier}: Proposed
    \code{group.com.tummycrypt.tcfs} for shared Keychain and container access.
  \item \textbf{Distribution}: \code{.dmg} with drag-to-install? Homebrew cask?
    Both?
  \item \textbf{Minimum macOS}: \code{NSFileProviderReplicatedExtension}
    matured in macOS 13+. Recommend targeting macOS 13 (Ventura) as minimum.
  \item \textbf{iOS timeline}: Shared crate enables iOS but requires Xcode
    project, TestFlight, and App Store submission. Defer to Phase 3+.
\end{enumerate}

\end{document}
