\documentclass[11pt,a4paper]{article}
% preamble.tex — Shared LaTeX preamble for all tcfs design documents
% Usage: % preamble.tex — Shared LaTeX preamble for all tcfs design documents
% Usage: % preamble.tex — Shared LaTeX preamble for all tcfs design documents
% Usage: \input{preamble} at top of each .tex file

% Encoding and fonts
\usepackage[T1]{fontenc}
\usepackage[utf8]{inputenc}
\usepackage{lmodern}
\usepackage{textcomp}

% Page geometry
\usepackage[margin=2.5cm]{geometry}

% Colors and hyperlinks
\usepackage[dvipsnames]{xcolor}
\usepackage[colorlinks=true,linkcolor=NavyBlue,urlcolor=NavyBlue,citecolor=NavyBlue]{hyperref}

% Code listings
\usepackage{listings}
\lstset{
  basicstyle=\ttfamily\small,
  breaklines=true,
  frame=single,
  backgroundcolor=\color{gray!10},
  xleftmargin=2em,
  framexleftmargin=1.5em,
  numbers=none,
  showstringspaces=false,
  columns=flexible,
  literate={←}{$\leftarrow$}1 {→}{$\rightarrow$}1 {⟶}{$\longrightarrow$}1,
}
\lstdefinelanguage{json}{
  string=[s]{"}{"},
  comment=[l]{//},
  morecomment=[s]{/*}{*/},
  literate=
    *{:}{{{\color{NavyBlue}:}}}{1}
    {,}{{{\color{NavyBlue},}}}{1}
    {\{}{{{\color{NavyBlue}\{}}}{1}
    {\}}{{{\color{NavyBlue}\}}}}{1}
    {[}{{{\color{NavyBlue}[}}}{1}
    {]}{{{\color{NavyBlue}]}}}{1},
}
\lstdefinelanguage{toml}{
  comment=[l]{\#},
  string=[b]",
  morestring=[b]',
}
\lstdefinelanguage{protobuf}{
  morekeywords={service,rpc,returns,message,string,uint32,uint64,int64,bool,bytes,repeated,optional,oneof,enum,map},
  morecomment=[l]{//},
  morestring=[b]",
}

% Tables
\usepackage{longtable}
\usepackage{booktabs}
\usepackage{array}
\usepackage{tabularx}

% Graphics
\usepackage{graphicx}
\usepackage{float}

% Better lists
\usepackage{enumitem}

% Header/footer
\usepackage{fancyhdr}
\pagestyle{fancy}
\fancyhf{}
\fancyfoot[C]{\thepage}
\renewcommand{\headrulewidth}{0.4pt}

% Tighter spacing
\setlength{\parindent}{0pt}
\setlength{\parskip}{6pt}

% Convenience macros
\newcommand{\tcfs}{\texttt{tcfs}}
\newcommand{\tcfsd}{\texttt{tcfsd}}
\newcommand{\code}[1]{\texttt{#1}}
\newcommand{\filepath}[1]{\texttt{#1}}
 at top of each .tex file

% Encoding and fonts
\usepackage[T1]{fontenc}
\usepackage[utf8]{inputenc}
\usepackage{lmodern}
\usepackage{textcomp}

% Page geometry
\usepackage[margin=2.5cm]{geometry}

% Colors and hyperlinks
\usepackage[dvipsnames]{xcolor}
\usepackage[colorlinks=true,linkcolor=NavyBlue,urlcolor=NavyBlue,citecolor=NavyBlue]{hyperref}

% Code listings
\usepackage{listings}
\lstset{
  basicstyle=\ttfamily\small,
  breaklines=true,
  frame=single,
  backgroundcolor=\color{gray!10},
  xleftmargin=2em,
  framexleftmargin=1.5em,
  numbers=none,
  showstringspaces=false,
  columns=flexible,
  literate={←}{$\leftarrow$}1 {→}{$\rightarrow$}1 {⟶}{$\longrightarrow$}1,
}
\lstdefinelanguage{json}{
  string=[s]{"}{"},
  comment=[l]{//},
  morecomment=[s]{/*}{*/},
  literate=
    *{:}{{{\color{NavyBlue}:}}}{1}
    {,}{{{\color{NavyBlue},}}}{1}
    {\{}{{{\color{NavyBlue}\{}}}{1}
    {\}}{{{\color{NavyBlue}\}}}}{1}
    {[}{{{\color{NavyBlue}[}}}{1}
    {]}{{{\color{NavyBlue}]}}}{1},
}
\lstdefinelanguage{toml}{
  comment=[l]{\#},
  string=[b]",
  morestring=[b]',
}
\lstdefinelanguage{protobuf}{
  morekeywords={service,rpc,returns,message,string,uint32,uint64,int64,bool,bytes,repeated,optional,oneof,enum,map},
  morecomment=[l]{//},
  morestring=[b]",
}

% Tables
\usepackage{longtable}
\usepackage{booktabs}
\usepackage{array}
\usepackage{tabularx}

% Graphics
\usepackage{graphicx}
\usepackage{float}

% Better lists
\usepackage{enumitem}

% Header/footer
\usepackage{fancyhdr}
\pagestyle{fancy}
\fancyhf{}
\fancyfoot[C]{\thepage}
\renewcommand{\headrulewidth}{0.4pt}

% Tighter spacing
\setlength{\parindent}{0pt}
\setlength{\parskip}{6pt}

% Convenience macros
\newcommand{\tcfs}{\texttt{tcfs}}
\newcommand{\tcfsd}{\texttt{tcfsd}}
\newcommand{\code}[1]{\texttt{#1}}
\newcommand{\filepath}[1]{\texttt{#1}}
 at top of each .tex file

% Encoding and fonts
\usepackage[T1]{fontenc}
\usepackage[utf8]{inputenc}
\usepackage{lmodern}
\usepackage{textcomp}

% Page geometry
\usepackage[margin=2.5cm]{geometry}

% Colors and hyperlinks
\usepackage[dvipsnames]{xcolor}
\usepackage[colorlinks=true,linkcolor=NavyBlue,urlcolor=NavyBlue,citecolor=NavyBlue]{hyperref}

% Code listings
\usepackage{listings}
\lstset{
  basicstyle=\ttfamily\small,
  breaklines=true,
  frame=single,
  backgroundcolor=\color{gray!10},
  xleftmargin=2em,
  framexleftmargin=1.5em,
  numbers=none,
  showstringspaces=false,
  columns=flexible,
  literate={←}{$\leftarrow$}1 {→}{$\rightarrow$}1 {⟶}{$\longrightarrow$}1,
}
\lstdefinelanguage{json}{
  string=[s]{"}{"},
  comment=[l]{//},
  morecomment=[s]{/*}{*/},
  literate=
    *{:}{{{\color{NavyBlue}:}}}{1}
    {,}{{{\color{NavyBlue},}}}{1}
    {\{}{{{\color{NavyBlue}\{}}}{1}
    {\}}{{{\color{NavyBlue}\}}}}{1}
    {[}{{{\color{NavyBlue}[}}}{1}
    {]}{{{\color{NavyBlue}]}}}{1},
}
\lstdefinelanguage{toml}{
  comment=[l]{\#},
  string=[b]",
  morestring=[b]',
}
\lstdefinelanguage{protobuf}{
  morekeywords={service,rpc,returns,message,string,uint32,uint64,int64,bool,bytes,repeated,optional,oneof,enum,map},
  morecomment=[l]{//},
  morestring=[b]",
}

% Tables
\usepackage{longtable}
\usepackage{booktabs}
\usepackage{array}
\usepackage{tabularx}

% Graphics
\usepackage{graphicx}
\usepackage{float}

% Better lists
\usepackage{enumitem}

% Header/footer
\usepackage{fancyhdr}
\pagestyle{fancy}
\fancyhf{}
\fancyfoot[C]{\thepage}
\renewcommand{\headrulewidth}{0.4pt}

% Tighter spacing
\setlength{\parindent}{0pt}
\setlength{\parskip}{6pt}

% Convenience macros
\newcommand{\tcfs}{\texttt{tcfs}}
\newcommand{\tcfsd}{\texttt{tcfsd}}
\newcommand{\code}[1]{\texttt{#1}}
\newcommand{\filepath}[1]{\texttt{#1}}


\fancyhead[L]{\small\textsc{tcfs}}
\fancyhead[R]{\small Architecture}

\title{tcfs Architecture}
\author{tcfs / TummyCrypt}
\date{February 2026}

\begin{document}
\maketitle
\tableofcontents
\newpage

% ─────────────────────────────────────────────────────────────────────────────
\section{Overview}

\tcfs{} (TummyCrypt FileSystem) is a FOSS, self-hosted, S3-first synchronization
system that replaces proprietary cloud storage clients (odrive, Dropbox, etc.)
with a FUSE-based mount that transparently hydrates files on demand.

\subsection{Design Goals}

\begin{enumerate}
  \item \textbf{Self-hosted first} --- all data stays on user-controlled infrastructure (SeaweedFS).
  \item \textbf{On-demand hydration} --- files appear locally as lightweight stubs;
        only downloaded when accessed.
  \item \textbf{End-to-end encryption} --- client-side XChaCha20-Poly1305 encryption
        before upload; storage operator never sees plaintext.
  \item \textbf{Content-addressed deduplication} --- FastCDC chunking with BLAKE3
        hashing eliminates redundant storage.
  \item \textbf{Cross-platform} --- Linux (FUSE3), macOS (FUSE-T), Windows (Cloud Files API).
  \item \textbf{Kubernetes-native backend} --- stateless sync workers scaled by KEDA,
        NATS JetStream for reliable task dispatch.
\end{enumerate}

% ─────────────────────────────────────────────────────────────────────────────
\section{System Architecture}

The system is divided into client-side and server-side components connected
via S3 API and NATS JetStream.

\subsection{Client Side}

The client runs on user machines and consists of:

\begin{description}
  \item[\tcfsd{} (daemon)] Long-running process exposing a gRPC service over a
    Unix domain socket. Manages FUSE mounts, credential loading, sync state,
    and Prometheus metrics. Reports readiness via \code{sd\_notify(READY=1)}.
  \item[\tcfs{} (CLI)] Command-line interface for push, pull, mount, unmount,
    sync-status, and unsync operations.
  \item[\code{tcfs-tui} (TUI)] Interactive terminal dashboard built with
    ratatui~0.30, connecting to the daemon via gRPC. Four tabs: Dashboard,
    Config, Mounts, Secrets.
  \item[\code{tcfs-mcp} (MCP server)] Model Context Protocol server for AI
    agent integration. Six tools over stdio JSON-RPC transport.
\end{description}

\subsection{Server Side (Kubernetes)}

\begin{description}
  \item[SeaweedFS cluster] Distributed blob storage. Three Raft masters for
    metadata consensus, volume servers on Drobo 5C for data.
  \item[NATS JetStream] Reliable task queue with two streams:
    \code{SYNC\_TASKS} and \code{HYDRATION\_EVENTS}.
  \item[Sync workers] Stateless \tcfsd{} instances running in
    \code{--mode=worker}. Horizontally scaled by KEDA based on NATS queue depth.
  \item[Prometheus + Grafana] Observability stack monitoring throughput,
    queue depth, and FUSE latency.
\end{description}

% ─────────────────────────────────────────────────────────────────────────────
\section{Crate Map}

All Rust code lives in \filepath{crates/}. The workspace contains 13 crates:

\begin{longtable}{llp{8cm}}
  \toprule
  \textbf{Crate} & \textbf{Type} & \textbf{Purpose} \\
  \midrule
  \endhead
  \code{tcfs-core}        & library & Shared types, config schema, protobuf defs \\
  \code{tcfs-crypto}      & library & XChaCha20-Poly1305, Argon2id KDF, HKDF, BIP-39 \\
  \code{tcfs-secrets}     & library & SOPS/age/KDBX credential chain \\
  \code{tcfs-storage}     & library & OpenDAL abstraction, SeaweedFS native API \\
  \code{tcfs-chunks}      & library & FastCDC chunking, BLAKE3 hashing, zstd compression \\
  \code{tcfs-sync}        & library & Sync engine, JSON state cache, NATS consumers \\
  \code{tcfs-fuse}        & library & Linux FUSE driver (fuse3 crate) \\
  \code{tcfs-cloudfilter} & library & Windows Cloud Files API (skeleton) \\
  \code{tcfs-sops}        & library & SOPS+age fleet secret propagation \\
  \code{tcfsd}            & binary  & Daemon: gRPC, FUSE, metrics, systemd \\
  \code{tcfs-cli}         & binary  & CLI: push, pull, mount, status, unsync \\
  \code{tcfs-tui}         & binary  & TUI: ratatui dashboard \\
  \code{tcfs-mcp}         & binary  & MCP server: 6 tools, stdio transport \\
  \bottomrule
\end{longtable}

% ─────────────────────────────────────────────────────────────────────────────
\section{Stub File Format}

Files not yet downloaded appear as \code{.tc} stubs. The stub is a JSON file
containing metadata needed to reconstruct the original:

\begin{lstlisting}[language=json]
{
  "version": 1,
  "file_id": "<BLAKE3 hash of original file>",
  "original_name": "photo.jpg",
  "original_size": 4194304,
  "mime_type": "image/jpeg",
  "modified_at": "2026-02-20T12:00:00Z",
  "chunk_count": 3,
  "manifest_key": "chunks/manifests/<file_id>",
  "remote_prefix": "tcfs/default"
}
\end{lstlisting}

Directory stubs use the \code{.tcf} extension and list child entries. See the
Protocol Specification (docs/tex/protocol.tex) for the full format.

% ─────────────────────────────────────────────────────────────────────────────
\section{Hydration Sequence}

When a FUSE-mounted \code{.tc} stub is opened:

\begin{enumerate}
  \item FUSE \code{open()} intercepts the file access request.
  \item \tcfsd{} fetches the manifest from
        \filepath{\{prefix\}/manifests/\{file\_hash\}}.
  \item Chunks are fetched in parallel from
        \filepath{\{prefix\}/chunks/\{chunk\_hash\}}.
  \item Each chunk is decompressed (zstd), decrypted (XChaCha20-Poly1305),
        and verified (BLAKE3).
  \item Chunks are concatenated in order to reconstruct the original file.
  \item The \code{.tc} stub is atomically replaced with the hydrated file.
  \item \code{tcfs unsync <path>} converts it back to a stub.
\end{enumerate}

% ─────────────────────────────────────────────────────────────────────────────
\section{Credential Chain}

\tcfsd{} discovers age identities in order of precedence:

\begin{enumerate}
  \item \code{\$CREDENTIALS\_DIRECTORY/age-identity} --- systemd
        \code{LoadCredentialEncrypted}
  \item \code{\$SOPS\_AGE\_KEY\_FILE} --- path to an age key file
  \item \code{\$SOPS\_AGE\_KEY} --- literal age key content
  \item \filepath{\textasciitilde/.config/sops/age/keys.txt} --- default fallback
\end{enumerate}

Once loaded, the age identity decrypts SOPS-encrypted credential YAML files,
providing S3 access keys to the OpenDAL operator. An mtime watcher
auto-reloads credentials when the file changes on disk.

% ─────────────────────────────────────────────────────────────────────────────
\section{Phase Roadmap}

\begin{longtable}{clp{9cm}}
  \toprule
  \textbf{Phase} & \textbf{Status} & \textbf{Scope} \\
  \midrule
  \endhead
  0 & Complete & Repo foundation, SOPS migration, Rust workspace stubs \\
  1 & Complete & Core daemon + secrets + gRPC \\
  2 & Complete & Sync engine + chunking + NATS \\
  3 & Complete & FUSE driver + \code{.tc} stubs + hydration \\
  4 & Complete & K8s backend + HPA + full OpenTofu deploy \\
  5 & Complete & Release pipeline + packaging + docs site \\
  6 & Complete & macOS FUSE-T + Windows CFAPI skeleton + GitLab mirror \\
  7 & Pending  & Production hardening + chaos tests \\
  \bottomrule
\end{longtable}

% ─────────────────────────────────────────────────────────────────────────────
\section{Infrastructure}

\subsection{Local Network (Bare-Metal)}

\begin{longtable}{ll}
  \toprule
  \textbf{Role} & \textbf{Address} \\
  \midrule
  \endhead
  SeaweedFS master-1         & 192.168.101.249:9333 \\
  SeaweedFS master-2         & 192.168.101.184:9333 \\
  SeaweedFS master-3         & 192.168.101.248:9333 \\
  Volume server (Drobo 5C)   & 192.168.101.171:8080 \\
  Filer / S3 gateway         & 192.168.101.146:8333 \\
  \bottomrule
\end{longtable}

\subsection{Civo Kubernetes}

\begin{longtable}{ll}
  \toprule
  \textbf{Property} & \textbf{Value} \\
  \midrule
  \endhead
  Cluster          & bitter-darkness-16657317 \\
  Namespace        & tcfs \\
  In-cluster SeaweedFS & seaweedfs.tcfs.svc.cluster.local:8333 \\
  In-cluster NATS  & nats.tcfs.svc.cluster.local:4222 \\
  Container image  & ghcr.io/tinyland-inc/tcfsd \\
  \bottomrule
\end{longtable}

\end{document}
