\documentclass[11pt,a4paper]{article}
% preamble.tex — Shared LaTeX preamble for all tcfs design documents
% Usage: % preamble.tex — Shared LaTeX preamble for all tcfs design documents
% Usage: % preamble.tex — Shared LaTeX preamble for all tcfs design documents
% Usage: \input{preamble} at top of each .tex file

% Encoding and fonts
\usepackage[T1]{fontenc}
\usepackage[utf8]{inputenc}
\usepackage{lmodern}
\usepackage{textcomp}

% Page geometry
\usepackage[margin=2.5cm]{geometry}

% Colors and hyperlinks
\usepackage[dvipsnames]{xcolor}
\usepackage[colorlinks=true,linkcolor=NavyBlue,urlcolor=NavyBlue,citecolor=NavyBlue]{hyperref}

% Code listings
\usepackage{listings}
\lstset{
  basicstyle=\ttfamily\small,
  breaklines=true,
  frame=single,
  backgroundcolor=\color{gray!10},
  xleftmargin=2em,
  framexleftmargin=1.5em,
  numbers=none,
  showstringspaces=false,
  columns=flexible,
  literate={←}{$\leftarrow$}1 {→}{$\rightarrow$}1 {⟶}{$\longrightarrow$}1,
}
\lstdefinelanguage{json}{
  string=[s]{"}{"},
  comment=[l]{//},
  morecomment=[s]{/*}{*/},
  literate=
    *{:}{{{\color{NavyBlue}:}}}{1}
    {,}{{{\color{NavyBlue},}}}{1}
    {\{}{{{\color{NavyBlue}\{}}}{1}
    {\}}{{{\color{NavyBlue}\}}}}{1}
    {[}{{{\color{NavyBlue}[}}}{1}
    {]}{{{\color{NavyBlue}]}}}{1},
}
\lstdefinelanguage{toml}{
  comment=[l]{\#},
  string=[b]",
  morestring=[b]',
}
\lstdefinelanguage{protobuf}{
  morekeywords={service,rpc,returns,message,string,uint32,uint64,int64,bool,bytes,repeated,optional,oneof,enum,map},
  morecomment=[l]{//},
  morestring=[b]",
}

% Tables
\usepackage{longtable}
\usepackage{booktabs}
\usepackage{array}
\usepackage{tabularx}

% Graphics
\usepackage{graphicx}
\usepackage{float}

% Better lists
\usepackage{enumitem}

% Header/footer
\usepackage{fancyhdr}
\pagestyle{fancy}
\fancyhf{}
\fancyfoot[C]{\thepage}
\renewcommand{\headrulewidth}{0.4pt}

% Tighter spacing
\setlength{\parindent}{0pt}
\setlength{\parskip}{6pt}

% Convenience macros
\newcommand{\tcfs}{\texttt{tcfs}}
\newcommand{\tcfsd}{\texttt{tcfsd}}
\newcommand{\code}[1]{\texttt{#1}}
\newcommand{\filepath}[1]{\texttt{#1}}
 at top of each .tex file

% Encoding and fonts
\usepackage[T1]{fontenc}
\usepackage[utf8]{inputenc}
\usepackage{lmodern}
\usepackage{textcomp}

% Page geometry
\usepackage[margin=2.5cm]{geometry}

% Colors and hyperlinks
\usepackage[dvipsnames]{xcolor}
\usepackage[colorlinks=true,linkcolor=NavyBlue,urlcolor=NavyBlue,citecolor=NavyBlue]{hyperref}

% Code listings
\usepackage{listings}
\lstset{
  basicstyle=\ttfamily\small,
  breaklines=true,
  frame=single,
  backgroundcolor=\color{gray!10},
  xleftmargin=2em,
  framexleftmargin=1.5em,
  numbers=none,
  showstringspaces=false,
  columns=flexible,
  literate={←}{$\leftarrow$}1 {→}{$\rightarrow$}1 {⟶}{$\longrightarrow$}1,
}
\lstdefinelanguage{json}{
  string=[s]{"}{"},
  comment=[l]{//},
  morecomment=[s]{/*}{*/},
  literate=
    *{:}{{{\color{NavyBlue}:}}}{1}
    {,}{{{\color{NavyBlue},}}}{1}
    {\{}{{{\color{NavyBlue}\{}}}{1}
    {\}}{{{\color{NavyBlue}\}}}}{1}
    {[}{{{\color{NavyBlue}[}}}{1}
    {]}{{{\color{NavyBlue}]}}}{1},
}
\lstdefinelanguage{toml}{
  comment=[l]{\#},
  string=[b]",
  morestring=[b]',
}
\lstdefinelanguage{protobuf}{
  morekeywords={service,rpc,returns,message,string,uint32,uint64,int64,bool,bytes,repeated,optional,oneof,enum,map},
  morecomment=[l]{//},
  morestring=[b]",
}

% Tables
\usepackage{longtable}
\usepackage{booktabs}
\usepackage{array}
\usepackage{tabularx}

% Graphics
\usepackage{graphicx}
\usepackage{float}

% Better lists
\usepackage{enumitem}

% Header/footer
\usepackage{fancyhdr}
\pagestyle{fancy}
\fancyhf{}
\fancyfoot[C]{\thepage}
\renewcommand{\headrulewidth}{0.4pt}

% Tighter spacing
\setlength{\parindent}{0pt}
\setlength{\parskip}{6pt}

% Convenience macros
\newcommand{\tcfs}{\texttt{tcfs}}
\newcommand{\tcfsd}{\texttt{tcfsd}}
\newcommand{\code}[1]{\texttt{#1}}
\newcommand{\filepath}[1]{\texttt{#1}}
 at top of each .tex file

% Encoding and fonts
\usepackage[T1]{fontenc}
\usepackage[utf8]{inputenc}
\usepackage{lmodern}
\usepackage{textcomp}

% Page geometry
\usepackage[margin=2.5cm]{geometry}

% Colors and hyperlinks
\usepackage[dvipsnames]{xcolor}
\usepackage[colorlinks=true,linkcolor=NavyBlue,urlcolor=NavyBlue,citecolor=NavyBlue]{hyperref}

% Code listings
\usepackage{listings}
\lstset{
  basicstyle=\ttfamily\small,
  breaklines=true,
  frame=single,
  backgroundcolor=\color{gray!10},
  xleftmargin=2em,
  framexleftmargin=1.5em,
  numbers=none,
  showstringspaces=false,
  columns=flexible,
  literate={←}{$\leftarrow$}1 {→}{$\rightarrow$}1 {⟶}{$\longrightarrow$}1,
}
\lstdefinelanguage{json}{
  string=[s]{"}{"},
  comment=[l]{//},
  morecomment=[s]{/*}{*/},
  literate=
    *{:}{{{\color{NavyBlue}:}}}{1}
    {,}{{{\color{NavyBlue},}}}{1}
    {\{}{{{\color{NavyBlue}\{}}}{1}
    {\}}{{{\color{NavyBlue}\}}}}{1}
    {[}{{{\color{NavyBlue}[}}}{1}
    {]}{{{\color{NavyBlue}]}}}{1},
}
\lstdefinelanguage{toml}{
  comment=[l]{\#},
  string=[b]",
  morestring=[b]',
}
\lstdefinelanguage{protobuf}{
  morekeywords={service,rpc,returns,message,string,uint32,uint64,int64,bool,bytes,repeated,optional,oneof,enum,map},
  morecomment=[l]{//},
  morestring=[b]",
}

% Tables
\usepackage{longtable}
\usepackage{booktabs}
\usepackage{array}
\usepackage{tabularx}

% Graphics
\usepackage{graphicx}
\usepackage{float}

% Better lists
\usepackage{enumitem}

% Header/footer
\usepackage{fancyhdr}
\pagestyle{fancy}
\fancyhf{}
\fancyfoot[C]{\thepage}
\renewcommand{\headrulewidth}{0.4pt}

% Tighter spacing
\setlength{\parindent}{0pt}
\setlength{\parskip}{6pt}

% Convenience macros
\newcommand{\tcfs}{\texttt{tcfs}}
\newcommand{\tcfsd}{\texttt{tcfsd}}
\newcommand{\code}[1]{\texttt{#1}}
\newcommand{\filepath}[1]{\texttt{#1}}


\fancyhead[L]{\small\textsc{tcfs}}
\fancyhead[R]{\small Security Model}

\title{tcfs Security Model}
\author{tcfs / TummyCrypt}
\date{February 2026}

\begin{document}
\maketitle
\tableofcontents
\newpage

% ─────────────────────────────────────────────────────────────────────────────
\section{Threat Model}

\tcfs{} protects user data stored on self-hosted SeaweedFS. The primary threats
are:

\begin{enumerate}
  \item \textbf{Storage compromise} --- attacker gains read access to SeaweedFS
        volumes.
  \item \textbf{Network interception} --- attacker observes S3/NATS traffic.
  \item \textbf{Client compromise} --- attacker gains access to an enrolled device.
  \item \textbf{Credential theft} --- attacker obtains S3 access keys.
\end{enumerate}

\subsection{What tcfs Protects Against}

\begin{description}
  \item[At-rest confidentiality] Files are encrypted client-side before upload
    using XChaCha20-Poly1305. The storage operator never sees plaintext.
  \item[Credential exposure in memory] S3 secret keys are held in
    \code{SecretString} (zeroized on drop).
  \item[Credential exposure on disk] Credentials are stored in SOPS-encrypted
    YAML, decrypted at runtime with age identities.
  \item[Password visibility] KDBX passwords are read via \code{rpassword},
    never passed as CLI arguments (preventing \code{ps} exposure).
\end{description}

\subsection{What tcfs Does NOT Protect Against}

\begin{description}
  \item[Key extraction] A root-level attacker with ptrace access can read
    decrypted keys from process memory.
  \item[Compromised devices] If a device's age private key is extracted,
    an attacker can decrypt all files accessible to that device.
  \item[Denial of service] \tcfs{} does not defend against data deletion or
    corruption on SeaweedFS.
\end{description}

% ─────────────────────────────────────────────────────────────────────────────
\section{Encryption Architecture}

\subsection{Key Hierarchy}

\begin{description}
  \item[User Master Key] 256-bit, derived via Argon2id from user passphrase.
    Root of all derived keys.
  \item[File Encryption Key] Per-file, 256-bit random. Wrapped by master key
    using XChaCha20-Poly1305. Used for chunk encryption.
  \item[Manifest Encryption Key] Derived via HKDF-SHA256 from master key.
    Encrypts file manifests (chunk lists, metadata).
  \item[Name Encryption Key] Derived via HKDF-SHA256 from master key.
    AES-SIV deterministic filename encryption.
\end{description}

\subsection{Chunk Encryption}

Each file chunk is independently encrypted:

\begin{enumerate}
  \item Plaintext chunk is compressed with zstd (level 3).
  \item Compressed data is encrypted with XChaCha20-Poly1305:
    \begin{itemize}
      \item Key: file encryption key (256-bit)
      \item Nonce: 192-bit random (no nonce management required)
      \item AAD: \code{chunk\_index} (8 bytes BE) $\|$ \code{file\_id} (32 bytes)
    \end{itemize}
  \item BLAKE3 hash of the ciphertext becomes the CAS key.
  \item Ciphertext is uploaded to SeaweedFS.
\end{enumerate}

Chunk format on disk/wire:

\begin{lstlisting}
[24 bytes: nonce][N bytes: ciphertext][16 bytes: Poly1305 tag]
\end{lstlisting}

\subsection{Recovery Key}

A BIP-39 24-word mnemonic is generated during \code{tcfs init}. This mnemonic
can regenerate the master key independently of the passphrase. Users must
store it offline in a secure location.

% ─────────────────────────────────────────────────────────────────────────────
\section{Credential Management}

\subsection{Discovery Chain}

Credentials are discovered in order of precedence:

\begin{enumerate}
  \item \code{\$CREDENTIALS\_DIRECTORY/age-identity} --- systemd
        \code{LoadCredentialEncrypted}
  \item \code{\$SOPS\_AGE\_KEY\_FILE} --- path to an age key file
  \item \code{\$SOPS\_AGE\_KEY} --- literal age key content
  \item \filepath{\textasciitilde/.config/sops/age/keys.txt} --- default fallback
\end{enumerate}

\subsection{SOPS-Encrypted Files}

S3 credentials are stored in SOPS-encrypted YAML files. At runtime, \tcfsd{}
decrypts these files using the discovered age identity, extracting
\code{access\_key\_id} and \code{secret\_access\_key} for the OpenDAL S3
operator.

\subsection{Memory Safety}

\begin{itemize}
  \item \code{secret\_access\_key} stored as \code{secrecy::SecretString}
        (zeroized on drop).
  \item Temporary plaintext copies explicitly \code{zeroize()}'d after use.
  \item Debug formatting redacts secrets:
        \code{S3Credentials \{ secret\_access\_key: "[REDACTED]" \}}.
\end{itemize}

% ─────────────────────────────────────────────────────────────────────────────
\section{Credential Rotation}

\subsection{Automated Rotation}

\tcfsd{} watches the SOPS credential file for changes. When modified:

\begin{enumerate}
  \item A 500ms debounce coalesces rapid writes (e.g., atomic replace).
  \item The file is re-decrypted with the configured age identity.
  \item The shared credential store is atomically swapped.
  \item Existing S3 connections continue with old credentials until the next
        request.
\end{enumerate}

\subsection{Manual Rotation}

\begin{lstlisting}[language=bash]
# Interactive
tcfs rotate-credentials

# Non-interactive (reads from environment)
tcfs rotate-credentials --non-interactive
\end{lstlisting}

The non-interactive mode reads \code{AWS\_ACCESS\_KEY\_ID} and
\code{AWS\_SECRET\_ACCESS\_KEY} from the environment.

Each rotation creates a timestamped SOPS-encrypted backup:
\code{credentials.sops.yaml.bak.<epoch>}.

% ─────────────────────────────────────────────────────────────────────────────
\section{TLS Configuration}

\subsection{S3 (SeaweedFS)}

\begin{lstlisting}[language=toml]
[storage]
endpoint = "https://seaweedfs.example.com:8333"
enforce_tls = true
ca_cert_path = "/etc/tcfs/certs/ca.pem"
\end{lstlisting}

When \code{enforce\_tls = true}, HTTP endpoints produce an error at startup.
When \code{false} (default), HTTP endpoints produce a warning.

\subsection{NATS}

\begin{lstlisting}[language=toml]
[sync]
nats_url = "tls://nats.example.com:4222"
nats_tls = true
nats_ca_cert = "/etc/tcfs/certs/nats-ca.pem"
\end{lstlisting}

% ─────────────────────────────────────────────────────────────────────────────
\section{Device Identity}

Each enrolled device has:

\begin{itemize}
  \item An age X25519 keypair (generated during \code{tcfs init}).
  \item A device registry entry (\filepath{\textasciitilde/.local/share/tcfs/devices.json}).
  \item Platform keychain storage for session keys (macOS Keychain,
        GNOME Keyring, Windows Credential Manager).
\end{itemize}

\subsection{Revoking a Device}

\begin{lstlisting}[language=bash]
tcfs device revoke <device-name>
\end{lstlisting}

Marks the device as revoked in the registry. Future versions will implement
key re-encryption to exclude revoked devices from new file keys.

% ─────────────────────────────────────────────────────────────────────────────
\section{Config File Security}

When \code{config\_file\_mode\_check = true} (default), \tcfsd{} warns at
startup if the configuration file is world-readable.

\begin{lstlisting}[language=bash]
chmod 600 /etc/tcfs/config.toml
chmod 600 /etc/tcfs/credentials.sops.yaml
chown tcfs:tcfs /etc/tcfs/*
\end{lstlisting}

% ─────────────────────────────────────────────────────────────────────────────
\section{Reporting Security Issues}

Report security vulnerabilities to: \href{mailto:jess@sulliwood.org}{jess@sulliwood.org}.

\begin{itemize}
  \item Description of the vulnerability
  \item Steps to reproduce
  \item Potential impact
  \item Suggested fix (if any)
\end{itemize}

We aim to acknowledge reports within 48 hours and provide a fix within 7 days
for critical issues.

\end{document}
